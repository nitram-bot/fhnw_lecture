%% Generated by Sphinx.
\def\sphinxdocclass{jupyterBook}
\documentclass[letterpaper,10pt,english]{jupyterBook}
\ifdefined\pdfpxdimen
   \let\sphinxpxdimen\pdfpxdimen\else\newdimen\sphinxpxdimen
\fi \sphinxpxdimen=.75bp\relax
%% turn off hyperref patch of \index as sphinx.xdy xindy module takes care of
%% suitable \hyperpage mark-up, working around hyperref-xindy incompatibility
\PassOptionsToPackage{hyperindex=false}{hyperref}
%% memoir class requires extra handling
\makeatletter\@ifclassloaded{memoir}
{\ifdefined\memhyperindexfalse\memhyperindexfalse\fi}{}\makeatother

\PassOptionsToPackage{warn}{textcomp}

\catcode`^^^^00a0\active\protected\def^^^^00a0{\leavevmode\nobreak\ }
\usepackage{cmap}
\usepackage{fontspec}
\defaultfontfeatures[\rmfamily,\sffamily,\ttfamily]{}
\usepackage{amsmath,amssymb,amstext}
\usepackage{polyglossia}
\setmainlanguage{english}



\setmainfont{FreeSerif}[
  Extension      = .otf,
  UprightFont    = *,
  ItalicFont     = *Italic,
  BoldFont       = *Bold,
  BoldItalicFont = *BoldItalic
]
\setsansfont{FreeSans}[
  Extension      = .otf,
  UprightFont    = *,
  ItalicFont     = *Oblique,
  BoldFont       = *Bold,
  BoldItalicFont = *BoldOblique,
]
\setmonofont{FreeMono}[
  Extension      = .otf,
  UprightFont    = *,
  ItalicFont     = *Oblique,
  BoldFont       = *Bold,
  BoldItalicFont = *BoldOblique,
]


\usepackage[Bjarne]{fncychap}
\usepackage[,numfigreset=1,mathnumfig]{sphinx}

\fvset{fontsize=\small}
\usepackage{geometry}


% Include hyperref last.
\usepackage{hyperref}
% Fix anchor placement for figures with captions.
\usepackage{hypcap}% it must be loaded after hyperref.
% Set up styles of URL: it should be placed after hyperref.
\urlstyle{same}


\usepackage{sphinxmessages}



         \usepackage[Latin,Greek]{ucharclasses}
        \usepackage{unicode-math}
        % fixing title of the toc
        \addto\captionsenglish{\renewcommand{\contentsname}{Contents}}
        

\title{My sample book}
\date{Jul 13, 2021}
\release{}
\author{The Jupyter Book Community}
\newcommand{\sphinxlogo}{\vbox{}}
\renewcommand{\releasename}{}
\makeindex
\begin{document}

\pagestyle{empty}
\sphinxmaketitle
\pagestyle{plain}
\sphinxtableofcontents
\pagestyle{normal}
\phantomsection\label{\detokenize{intro::doc}}


\sphinxAtStartPar
This is a small sample book to give you a feel for how book content is
structured.

\begin{sphinxadmonition}{note}{Note:}
\sphinxAtStartPar
Here is a note!
\end{sphinxadmonition}

\sphinxAtStartPar
And here is a code block:

\begin{sphinxVerbatim}[commandchars=\\\{\}]
\PYG{n}{e} \PYG{o}{=} \PYG{n}{mc}\PYG{o}{\PYGZca{}}\PYG{l+m+mi}{2}
\end{sphinxVerbatim}

\sphinxAtStartPar
Check out the content pages bundled with this sample book to see more.


\chapter{Markdown Files}
\label{\detokenize{markdown:markdown-files}}\label{\detokenize{markdown::doc}}
\sphinxAtStartPar
Whether you write your book’s content in Jupyter Notebooks (\sphinxcode{\sphinxupquote{.ipynb}}) or
in regular markdown files (\sphinxcode{\sphinxupquote{.md}}), you’ll write in the same flavor of markdown
called \sphinxstylestrong{MyST Markdown}.


\section{What is MyST?}
\label{\detokenize{markdown:what-is-myst}}
\sphinxAtStartPar
MyST stands for “Markedly Structured Text”. It
is a slight variation on a flavor of markdown called “CommonMark” markdown,
with small syntax extensions to allow you to write \sphinxstylestrong{roles} and \sphinxstylestrong{directives}
in the Sphinx ecosystem.


\section{What are roles and directives?}
\label{\detokenize{markdown:what-are-roles-and-directives}}
\sphinxAtStartPar
Roles and directives are two of the most powerful tools in Jupyter Book. They
are kind of like functions, but written in a markup language. They both
serve a similar purpose, but \sphinxstylestrong{roles are written in one line}, whereas
\sphinxstylestrong{directives span many lines}. They both accept different kinds of inputs,
and what they do with those inputs depends on the specific role or directive
that is being called.


\subsection{Using a directive}
\label{\detokenize{markdown:using-a-directive}}
\sphinxAtStartPar
At its simplest, you can insert a directive into your book’s content like so:

\begin{sphinxVerbatim}[commandchars=\\\{\}]
```\PYGZob{}mydirectivename\PYGZcb{}
My directive content
```
\end{sphinxVerbatim}

\sphinxAtStartPar
This will only work if a directive with name \sphinxcode{\sphinxupquote{mydirectivename}} already exists
(which it doesn’t). There are many pre\sphinxhyphen{}defined directives associated with
Jupyter Book. For example, to insert a note box into your content, you can
use the following directive:

\begin{sphinxVerbatim}[commandchars=\\\{\}]
```\PYGZob{}note\PYGZcb{}
Here is a note
```
\end{sphinxVerbatim}

\sphinxAtStartPar
This results in:

\begin{sphinxadmonition}{note}{Note:}
\sphinxAtStartPar
Here is a note
\end{sphinxadmonition}

\sphinxAtStartPar
In your built book.

\sphinxAtStartPar
For more information on writing directives, see the
\sphinxhref{https://myst-parser.readthedocs.io/}{MyST documentation}.


\subsection{Using a role}
\label{\detokenize{markdown:using-a-role}}
\sphinxAtStartPar
Roles are very similar to directives, but they are less\sphinxhyphen{}complex and written
entirely on one line. You can insert a role into your book’s content with
this pattern:

\begin{sphinxVerbatim}[commandchars=\\\{\}]
Some content \PYGZob{}rolename\PYGZcb{}`and here is my role\PYGZsq{}s content!`
\end{sphinxVerbatim}

\sphinxAtStartPar
Again, roles will only work if \sphinxcode{\sphinxupquote{rolename}} is a valid role’s name. For example,
the \sphinxcode{\sphinxupquote{doc}} role can be used to refer to another page in your book. You can
refer directly to another page by its relative path. For example, the
role syntax \sphinxcode{\sphinxupquote{\{doc\}`intro`}} will result in: {\hyperref[\detokenize{intro::doc}]{\sphinxcrossref{\DUrole{doc}{Welcome to your Jupyter Book}}}}.

\sphinxAtStartPar
For more information on writing roles, see the
\sphinxhref{https://myst-parser.readthedocs.io/}{MyST documentation}.


\subsection{Adding a citation}
\label{\detokenize{markdown:adding-a-citation}}
\sphinxAtStartPar
You can also cite references that are stored in a \sphinxcode{\sphinxupquote{bibtex}} file. For example,
the following syntax: \sphinxcode{\sphinxupquote{\{cite\}`holdgraf\_evidence\_2014`}} will render like
this: {[}\hyperlink{cite.markdown:id3}{HdHPK14}{]}.

\sphinxAtStartPar
Moreoever, you can insert a bibliography into your page with this syntax:
The \sphinxcode{\sphinxupquote{\{bibliography\}}} directive must be used for all the \sphinxcode{\sphinxupquote{\{cite\}}} roles to
render properly.
For example, if the references for your book are stored in \sphinxcode{\sphinxupquote{references.bib}},
then the bibliography is inserted with:

\begin{sphinxVerbatim}[commandchars=\\\{\}]
```\PYGZob{}bibliography\PYGZcb{}
```
\end{sphinxVerbatim}

\sphinxAtStartPar
Resulting in a rendered bibliography that looks like:

\sphinxAtStartPar



\subsection{Executing code in your markdown files}
\label{\detokenize{markdown:executing-code-in-your-markdown-files}}
\sphinxAtStartPar
If you’d like to include computational content inside these markdown files,
you can use MyST Markdown to define cells that will be executed when your
book is built. Jupyter Book uses \sphinxstyleemphasis{jupytext} to do this.

\sphinxAtStartPar
First, add Jupytext metadata to the file. For example, to add Jupytext metadata
to this markdown page, run this command:

\begin{sphinxVerbatim}[commandchars=\\\{\}]
\PYG{n}{jupyter}\PYG{o}{\PYGZhy{}}\PYG{n}{book} \PYG{n}{myst} \PYG{n}{init} \PYG{n}{markdown}\PYG{o}{.}\PYG{n}{md}
\end{sphinxVerbatim}

\sphinxAtStartPar
Once a markdown file has Jupytext metadata in it, you can add the following
directive to run the code at build time:

\begin{sphinxVerbatim}[commandchars=\\\{\}]
```\PYGZob{}code\PYGZhy{}cell\PYGZcb{}
print(\PYGZdq{}Here is some code to execute\PYGZdq{})
```
\end{sphinxVerbatim}

\sphinxAtStartPar
When your book is built, the contents of any \sphinxcode{\sphinxupquote{\{code\sphinxhyphen{}cell\}}} blocks will be
executed with your default Jupyter kernel, and their outputs will be displayed
in\sphinxhyphen{}line with the rest of your content.

\sphinxAtStartPar
For more information about executing computational content with Jupyter Book,
see \sphinxhref{https://myst-nb.readthedocs.io/}{The MyST\sphinxhyphen{}NB documentation}.


\chapter{Content with notebooks}
\label{\detokenize{notebooks:content-with-notebooks}}\label{\detokenize{notebooks::doc}}
\sphinxAtStartPar
You can also create content with Jupyter Notebooks. This means that you can include
code blocks and their outputs in your book.


\section{Markdown + notebooks}
\label{\detokenize{notebooks:markdown-notebooks}}
\sphinxAtStartPar
As it is markdown, you can embed images, HTML, etc into your posts!

\sphinxAtStartPar


\sphinxAtStartPar
You can also \(add_{math}\) and
\begin{equation*}
\begin{split}
math^{blocks}
\end{split}
\end{equation*}
\sphinxAtStartPar
or
\begin{equation*}
\begin{split}
\begin{aligned}
\mbox{mean} la_{tex} \\ \\
math blocks
\end{aligned}
\end{split}
\end{equation*}
\sphinxAtStartPar
But make sure you \$Escape \$your \$dollar signs \$you want to keep!


\section{MyST markdown}
\label{\detokenize{notebooks:myst-markdown}}
\sphinxAtStartPar
MyST markdown works in Jupyter Notebooks as well. For more information about MyST markdown, check
out \sphinxhref{https://jupyterbook.org/content/myst.html}{the MyST guide in Jupyter Book},
or see \sphinxhref{https://myst-parser.readthedocs.io/en/latest/}{the MyST markdown documentation}.


\section{Code blocks and outputs}
\label{\detokenize{notebooks:code-blocks-and-outputs}}
\sphinxAtStartPar
Jupyter Book will also embed your code blocks and output in your book.
For example, here’s some sample Matplotlib code:

\begin{sphinxVerbatim}[commandchars=\\\{\}]
\PYG{k+kn}{from} \PYG{n+nn}{matplotlib} \PYG{k+kn}{import} \PYG{n}{rcParams}\PYG{p}{,} \PYG{n}{cycler}
\PYG{k+kn}{import} \PYG{n+nn}{matplotlib}\PYG{n+nn}{.}\PYG{n+nn}{pyplot} \PYG{k}{as} \PYG{n+nn}{plt}
\PYG{k+kn}{import} \PYG{n+nn}{numpy} \PYG{k}{as} \PYG{n+nn}{np}
\PYG{n}{plt}\PYG{o}{.}\PYG{n}{ion}\PYG{p}{(}\PYG{p}{)}
\end{sphinxVerbatim}

\begin{sphinxVerbatim}[commandchars=\\\{\}]
\PYGZlt{}matplotlib.pyplot.\PYGZus{}IonContext at 0x7f77b19f7750\PYGZgt{}
\end{sphinxVerbatim}

\noindent\sphinxincludegraphics{{notebooks_2_0}.png}

\sphinxAtStartPar
There is a lot more that you can do with outputs (such as including interactive outputs)
with your book. For more information about this, see \sphinxhref{https://jupyterbook.org}{the Jupyter Book documentation}

\begin{sphinxthebibliography}{HdHPK14}
\bibitem[HdHPK14]{markdown:id3}
\sphinxAtStartPar
Christopher Ramsay Holdgraf, Wendy de Heer, Brian N. Pasley, and Robert T. Knight. Evidence for Predictive Coding in Human Auditory Cortex. In \sphinxstyleemphasis{International Conference on Cognitive Neuroscience}. Brisbane, Australia, Australia, 2014. Frontiers in Neuroscience.
\end{sphinxthebibliography}







\renewcommand{\indexname}{Index}
\printindex
\end{document}